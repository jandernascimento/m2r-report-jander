
There have been several proposals for classification methods, all on behalf of scene understading. Due to the variety of sensors and problem to takle, the techniques varies. The techniques are adapted accordingly.

The work over this theme copes with classification by establishing some ground-rules, the constraints in which the method relies on to work.

\section{Grouping issues in robotics}

One important step in the scene understanding is the environment map building and locate the ego-robot inside this map. This problem is known as SLAM \cite{Leonard2002Mobile}, which stands for Simultaneous Localization and  Mapping.

SLAM is composed of several steps to reach its goal: landmark extraction, data association, state estimation, etc. Each of them can be solved using different techniques.

SLAM tackles one specific scenario. SLAM assumes static environment constraint, in another words, any moving object in the same environment as the robot will interfere in the mapping process. For this reason, there is another process called DATMO to deal with the dynamism of the environment.

DATMO stands for Detection and Tracking of Moving Objects. When dealing with mobile robots in a dynamic environment we apply the DATMO process to obtain the moving objects from the environment.

\section{Solutions for segmentation}

\subsection{Stereo-visio camera}

In the Henning Lategahn and Bernd Kitt work \cite{DBLP:conf/ivs/LategahnGHKE11}, they proposed a classification method using stereo vision. Their work was evaluated theoretically by simulation and pratically by running the method in real footage from a moving vehicle.

One of the test cases had shown the capability of classification between the static and dynamic environment. In some situations the observation time had to be longer in order to perform correctly the classification. One of the reasons for that is the Sequential Probability Ratio Test (SPRT) performed in the samples.

Although the goal is the same in their and our work - classification - our method uses only LIDAR and IMU information to perform such classification.

\subsection{Single mono-vision camera}

In the work developed by Migliore et al. \cite{Migliore_2009_ICRA} is a MONOSLAM..  ()

\subsection{LIDAR aproaches}

In the work \cite{4650636}, they use the Iterative Closest Point (ICP) \cite{10.1109/34.121791} to establish the displacement of the objects. 

This is difference in our method, first of all, we do not cluster to establish the boundaries of an object, this is not concerned by our application. For this reason we use grid based representation without clustering the objects, this was chosen due to it is parallelization and consequently it is speed.

We replaced the ICP algorithm by IMU measurements. The IMU measurements provide a approximate displacement of the vehicle, and based on that information we can calculate the displacement of the static objects in the scene.  
