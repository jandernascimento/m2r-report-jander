\section{Conclusion}
In this chapter we have given implementation details of motion detection modules with detailed algorithms of important functions. We have also presented some results of our work for real data obtained from demonstrator vehicle. We have seen that this technique works good especially when vehicle is moving on a straight road, relatively more false alarms appear when vehicle is turning. But because it is just low level classification of static and dynamic parts we believe that a high level module that will finally segment the objects and track them will be easily able to remove these false positives.

\section{Future Work}
Since this work has been carried out with an objective of using with Bayesian Occupancy Filter (BOF), so we plan to use the generated motion grid as an input to the BOF framework providing a priori information for moving parts. This will change the BOF behavior in the sense that instead of making an inference on velocities for all cells we will restrict this inference only for those regions detected as moving by our module. This restriction will not only improve the quality of results when later used by the tracking module but will also reduce the processing time for BOF.

\section{Acknowledgment}

I would like to thank Qadeer Baig for the continuous support and guidance during this work, including lunch time discussion on robotics subjects. Mathias Perrolaz for the opportunity to work in the E-Motion team, and for the productive brainstorms and insights. Thank both of them for the tough review on the draft version of this document. Last but not least, i would like to thank the team E-Motion itself, and the ones that are responsible for maintaning this group, turning the wheels in the backstage, and pushing us further. Thank you.